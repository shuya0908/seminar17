\documentclass{jsarticle}
% \documentclass[b4paper,landscape,14pt]{jsarticle}
\title{}
\author{}
\date{
% \number\year 年 \number\month 月
}
\usepackage{fenrir_v1_4_0}
\usepackage{ethm_v1_1_0}
\mathtoolsset{showonlyrefs=true}

\begin{document}
% \maketitle
\setcounter{section}{7}
\section{Stochastic Differential Equations}
\setcounter{subsection}{2}
\subsection{Solutions of Stochastic Differential Equations as Markov Processes}

コンパクトサポートを持つ $\real^d$ 上の二階連続的微分可能な実数値関数全体の集合を $C_c^2(\mathbf{R}^d)$ と表す:
\begin{align}
    C_c^2(\mathbf{R}^d)
    := \{f\in C^2(\mathbf{R}^d):\operatorname{supp}(f)\text{ はコンパクト}\}
\end{align}

\setcounter{thm}{6}
\begin{screen}
    \begin{thm}\label{thm:807}~
        \begin{enumerate}[label=(\arabic*)]
            \item
            半群 $(Q_t)_{t\ge0}$ は Feller 過程である.
            \item
            $(Q_t)_{t\ge0}$ の生成作用素 $L$ は
            \begin{align}
                C_c^2(\mathbf{R}^d)
                \subset D(L)
            \end{align}
            を満たし,任意の $f\in C_c^2(\mathbf{R}^d)$ に対して
            \begin{align}
                Lf(x)
                = \frac{1}{2}\sum_{i=1}^{d}\sum_{j=1}^{d}(\sigma\sigma^{\ast})_{ij}(x)\frac{\partial^{2}f}{\partial x_{i}\partial x_{j}}(x)
                + \sum_{i=1}^{d}b_{i}(x)\frac{\partial f}{\partial x_{i}}(x)
            \end{align}
            が成り立つ(ただし $\sigma^{\ast}$: $\sigma$ の転置行列).
        \end{enumerate}
    \end{thm}
\end{screen}

\end{document}
