\documentclass{jsarticle}
% \documentclass[b4paper,landscape,14pt]{jsarticle}
\title{}
\author{}
\date{
% \number\year 年 \number\month 月
}
\usepackage{fenrir_v1_4_0}
\usepackage{ethm_v1_1_0}
\mathtoolsset{showonlyrefs=true}

\begin{document}
% \maketitle
\setcounter{section}{7}
\section{Stochastic Differential Equations}
\setcounter{subsection}{2}
\subsection{Solutions of Stochastic Differential Equations as Markov Processes}

コンパクトサポートを持つ $\real^d$ 上の二階連続的微分可能な実数値関数全体の集合を $C_c^2(\mathbf{R}^d)$ と表す:
\begin{align}
    C_c^2(\mathbf{R}^d)
    := \{f\in C^2(\mathbf{R}^d):\operatorname{supp}(f)\text{ はコンパクト}\}
\end{align}

\setcounter{thm}{6}
\begin{screen}
    \begin{thm}\label{thm:807}~
        \begin{enumerate}[label=(\arabic*)]
            \item
            半群 $(Q_t)_{t\ge0}$ は Feller 過程である.
            \item
            $(Q_t)_{t\ge0}$ の生成作用素 $L$ は
            \begin{align}
                C_c^2(\mathbf{R}^d)
                \subset D(L)
            \end{align}
            を満たし,任意の $f\in C_c^2(\mathbf{R}^d)$ に対して
            \begin{align}
                Lf(x)
                = \frac{1}{2}\sum_{i=1}^{d}\sum_{j=1}^{d}(\sigma\sigma^{\ast})_{ij}(x)\frac{\partial^{2}f}{\partial x_{i}\partial x_{j}}(x)
                + \sum_{i=1}^{d}b_{i}(x)\frac{\partial f}{\partial x_{i}}(x)
            \end{align}
            が成り立つ(ただし $\sigma^{\ast}$: $\sigma$ の転置行列).
        \end{enumerate}
    \end{thm}
\end{screen}

\begin{proof}
    問題を単純にするため,$\sigma, b$ が有界の場合のみ示す(一般の場合?).
    \begin{enumerate}[label=(\arabic*)]
        \item
        示すべきことは
        \begin{enumerate}[label=(\roman*)]
            \item
            $\forall f\in C_{0}(\mathbf{R}^d), Q_{t}f\in C_{0}(\mathbf{R}^d)$
            \item
            $\forall f\in C_{0}(\mathbf{R}^d), \lVert Q_{t}f-f\rVert\xrightarrow{t\to0}0$
        \end{enumerate}
        の 2 つ.

        $f\in C_{0}(\mathbf{R}^d)$: 固定.
        \begin{enumerate}[label=(\roman*)]
            \item
            写像 $x\mapsto F_{x}(w)$ が連続であること,式 (8.2)
            \begin{align}
                Q_{t}f(x)
                = \int_{C(\mathbf{R_+}, \mathbf{R}^{m})}f(F_{x}(w)_{t})W(dw)
            \end{align}
            と優収束定理より $Q_{t}f$: 連続.
            \begin{align}
                X_{t}^{x}
                = x
                + \int_{0}^{t}\sigma(X_{s}^{x})dB_{s}
                + \int_{0}^{t}b(X_{s}^{x})ds
            \end{align}
            と書けることと,$\sigma, b$: 有界より
            \begin{align}
                E[(X_{t}^{x}-x)^2]
                &= E[(\int_{0}^{t}\sigma(X_{s}^{x})dB_{s}
                + \int_{0}^{t}b(X_{s}^{x})ds)^2] \\
                &\le 2(E[(\int_{0}^{t}\sigma(X_{s}^{x})dB_{s})^2]
                + E[(\int_{0}^{t}b(X_{s}^{x})ds)^2]) \\
                &\le 2(E[\int_{0}^{t}(\sigma(X_{s}^{x}))^{2}ds]
                + tE[\int_{0}^{t}(b(X_{s}^{x}))^{2}ds]) \\
                &\le 2(\lVert \sigma^2\rVert t+\lVert b^2\rVert t^2) \\
                &\le 2(\lVert \sigma^2\rVert\vee\lVert b^2\rVert)(t+t^2).
            \end{align}

            ここで $C:=2(\lVert \sigma^2\rVert\vee\lVert b^2\rVert)$ と定めると,これは $t, x$ に依らない定数で
            \begin{align}
                E[(X_{t}^{x}-x)^2]
                \le C(t+t^2).
                \label{eq:804}
            \end{align}
            \item
            
        \end{enumerate}
        
        \item
        
    \end{enumerate}
\end{proof}

\end{document}
