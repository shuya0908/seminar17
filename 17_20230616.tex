\documentclass{jsarticle}
% \documentclass[b4paper,landscape,14pt]{jsarticle}
\title{}
\author{}
\date{
% \number\year 年 \number\month 月
}
\usepackage{fenrir_v1_4_0}
\usepackage{ethm_v1_1_0}
\mathtoolsset{showonlyrefs=true}

\begin{document}
% \maketitle
\setcounter{section}{7}
\section{Stochastic Differential Equations}
\setcounter{subsection}{2}
\subsection{Solutions of Stochastic Differential Equations as Markov Processes}

コンパクトサポートを持つ $\real^d$ 上の二階連続的微分可能な実数値関数全体の集合を $C_c^2(\mathbf{R}^d)$ と表す:
\begin{align}
    C_c^2(\mathbf{R}^d)
    := \{f\in C^2(\mathbf{R}^d):\operatorname{supp}(f)\text{ はコンパクト}\}
\end{align}

\setcounter{thm}{6}
\begin{screen}
    \begin{thm}\label{thm:807}~
        \begin{enumerate}[label=(\arabic*)]
            \item
            半群 $(Q_t)_{t\ge0}$ は Feller 半群である.
            \item
            $(Q_t)_{t\ge0}$ の生成作用素 $L$ は
            \begin{align}
                C_c^2(\mathbf{R}^d)
                \subset D(L)
            \end{align}
            を満たし,任意の $f\in C_c^2(\mathbf{R}^d)$ に対して
            \begin{align}
                Lf(x)
                = \frac{1}{2}\sum_{i, j=1}^{d}(\sigma\sigma^{\ast})_{ij}(x)\frac{\partial^{2}f}{\partial x_{i}\partial x_{j}}(x)
                + \sum_{i=1}^{d}b_{i}(x)\frac{\partial f}{\partial x_{i}}(x)
            \end{align}
            が成り立つ(ただし $\sigma^{\ast}$: $\sigma$ の転置行列).
        \end{enumerate}
    \end{thm}
\end{screen}

\begin{proof}
    問題を単純にするため,$\sigma, b$ が有界の場合のみ示す(一般の場合?).
    \begin{enumerate}[label=(\arabic*)]
        \item
        示すべきことは
        \begin{enumerate}[label=(\roman*)]
            \item
            $\forall f\in C_{0}(\mathbf{R}^d), Q_{t}f\in C_{0}(\mathbf{R}^d)$
            \item
            $\forall f\in C_{0}(\mathbf{R}^d), \lVert Q_{t}f-f\rVert\xrightarrow{t\to0}0$
        \end{enumerate}
        の 2 つ.

        $f\in C_{0}(\mathbf{R}^d)$: 固定.
        \begin{enumerate}[label=(\roman*)]
            \item
            写像 $x\mapsto F_{x}(w)$ が連続であること,(8.2) 式
            \begin{align}
                Q_{t}f(x)
                = \int_{C(\mathbf{R_+}, \mathbf{R}^{m})}f(F_{x}(w)_{t})W(dw)
            \end{align}
            と優収束定理より $Q_{t}f$: 連続\nazo.
            
            \begin{screen}
                $\because)$
            \end{screen}
            \begin{align}
                X_{t}^{x}
                = x
                + \int_{0}^{t}\sigma(X_{s}^{x})dB_{s}
                + \int_{0}^{t}b(X_{s}^{x})ds
            \end{align}
            と書けることと,$\sigma, b$: 有界より
            \begin{align}
                E[(X_{t}^{x}-x)^2]
                &= E[(\int_{0}^{t}\sigma(X_{s}^{x})dB_{s}
                + \int_{0}^{t}b(X_{s}^{x})ds)^2] \\
                &\le 2(E[(\int_{0}^{t}\sigma(X_{s}^{x})dB_{s})^2]
                + E[(\int_{0}^{t}b(X_{s}^{x})ds)^2]) \\
                &\le 2(E[\int_{0}^{t}(\sigma(X_{s}^{x}))^{2}ds]
                + tE[\int_{0}^{t}(b(X_{s}^{x}))^{2}ds]) \\
                &\le 2(\lVert \sigma^2\rVert t+\lVert b^2\rVert t^2) \\
                &\le 2(\lVert \sigma^2\rVert\vee\lVert b^2\rVert)(t+t^2).
            \end{align}

            ここで $C:=2(\lVert \sigma^2\rVert\vee\lVert b^2\rVert)$ と定めると,これは $t, x$ に依らない定数で
            \setcounter{equation}{3}
            \begin{align}
                \sup_{x\in\mathbf{R}^d}E[(X_{t}^{x}-x)^2]
                \le C(t+t^2).
                \label{eq:804}
            \end{align}

            $\forall t\ge0$ に対し,Markov の不等式(Chebyshev の不等式の特別な場合)と \eqref{eq:804} より
            \begin{align}
                \sup_{x\in\mathbf{R}^d}P(\lvert X_{t}^{x}-x\rvert>A)
                &\le \frac{\sup_{x\in\mathbf{R}^d}E[(X_{t}^{x}-x)^2]}{A^2} \\
                &\le \frac{C(t+t^2)}{A^2}\xrightarrow{A\to\infty}0.
            \end{align}

            一方
            \begin{align}
                \lvert Q_{t}f(x)\rvert
                &= \lvert E[f(X_{t}^{x})]\rvert \\
                &\le \lvert E[f(X_{t}^{x})\bm{1}_{\{\lvert X_{t}^{x}-x\rvert\le A\}}]\rvert
                + \lvert E[f(X_{t}^{x})\bm{1}_{\{\lvert X_{t}^{x}-x\rvert>A\}}]\rvert \\
                &\le \lvert E[f(X_{t}^{x})\bm{1}_{\{\lvert X_{t}^{x}-x\rvert\le A\}}]\rvert
                + \lVert f\rVert P(\lvert X_{t}^{x}-x\rvert>A).
            \end{align}

            これと $f\in C_{0}(\mathbf{R}^d)$ より
            \begin{align}
                \limsup_{x\to\infty}\lvert Q_{t}f(x)\rvert
                &\le \limsup_{x\to\infty}\lvert E[f(X_{t}^{x})\bm{1}_{\{\lvert X_{t}^{x}-x\rvert\le A\}}]\rvert
                + \lVert f\rVert \limsup_{x\to\infty}P(\lvert X_{t}^{x}-x\rvert>A) \\
                &\le \lvert E[\limsup_{x\to\infty}f(X_{t}^{x})\bm{1}_{\{\lvert X_{t}^{x}-x\rvert\le A\}}]\rvert
                + \lVert f\rVert \sup_{x\in\mathbf{R}^d}P(\lvert X_{t}^{x}-x\rvert>A) \\
                &= \lVert f\rVert \sup_{x\in\mathbf{R}^d}P(\lvert X_{t}^{x}-x\rvert>A).
            \end{align}

            よって $A$ の任意性より $\limsup_{x\to\infty}\lvert Q_{t}f(x)\rvert=0.$
            したがって $\lim_{x\to\infty}\lvert Q_{t}f(x)\rvert=0$ より 
            \begin{align}
                \lim_{x\to\infty}Q_{t}f(x)=0.
            \end{align}
            
            以上より $Q_{t}f$ は $\mathbf{R}^d$ 上の実数値連続関数で $\lim_{x\to\infty}Q_{t}f(x)=0$ を満たすので $Q_{t}f\in C_{0}(\mathbf{R}^d).$
                
            \item
            $\forall \varepsilon>0$: 固定.
            \begin{align}
                \sup_{x\in\mathbf{R}^d}\lvert E[f(X_{t}^{x}-x)]-f(x)\rvert
                \le \sup_{x, y\in\mathbf{R}^d, \lvert x-y\rvert<\varepsilon}\lvert f(x)-f(y)\rvert
                + 2\lVert f\rVert\sup_{x\in\mathbf{R}^d}P(\lvert X_{t}^{x}-x\rvert>\varepsilon)\nazo.
            \end{align}
            
            \begin{screen}
                $\because)$
            \end{screen}

            ここで \eqref{eq:804} 及び Markov の不等式より
            \begin{align}
                \sup_{x\in\mathbf{R}^d}P(\lvert X_{t}^{x}-x\rvert>\varepsilon)
                \le \frac{C(t+t^2)}{\varepsilon^2}\xrightarrow{t\to0}0.
            \end{align}

            よって
            \begin{align}
                \limsup_{t\to0}\lVert Q_{t}f-f\rVert
                &= \limsup_{t\to0}(\sup_{x\in\mathbf{R}^d}\lvert E[f(X_{t}^{x}-x)]-f(x)\rvert) \\
                &\le \sup_{x, y\in\mathbf{R}^d, \lvert x-y\rvert<\varepsilon}\lvert f(x)-f(y)\rvert
            \end{align}
            となるが,$f$ の連続性より $\varepsilon$ を十分小さくとれば LHS $=0.$
            したがって $\lim_{t\to0}\lVert Q_{t}f-f\rVert.$
        \end{enumerate}

        以上より $(Q_{t})_{t\ge0}$ は Feller 半群であることが示された.
        
        \item
        $f\in C_{c}^2(\mathbf{R}^d)$ とする.
        $f(X_{t}^{x})=f(X_{t}^{x, 1},\dotsb, X_{t}^{x, d})$ に対し It\^{o}'s formula を適用すると
        \begin{align}
            f(X_{t}^{x})
            &= f(x)
            + \sum_{i=1}^{d}\int_{0}^{t}\frac{\partial f}{\partial x_{i}}(X_{s}^{x})dX_{s}^{x, i}
            + \frac{1}{2}\sum_{i, i'=1}^{d}\int_{0}^{t}\frac{\partial f^2}{\partial x_{i}\partial x_{i'}}(X_{s}^{x})d\langle X^{x, i}, X^{x, i'}\rangle_{s}.
        \end{align}

        ここで
        \begin{itemize}
            \item 
            $\displaystyle X_{t}^{x, i}
            = x_i
            + \sum_{j=1}^{m}\int_{0}^{t}\sigma_{ij}(X_{s}^{x})dB_{s}^{j}
            + \int_{0}^{t}b_{i}(X_{s}^{x})ds$ \\
            $\displaystyle \implies dX_{s}^{x, i}
            = \sum_{j=1}^{m}\sigma_{ij}(X_{s}^{x})dB_{s}^{j}
            + b_{i}(X_{s}^{x})ds$
            \item 
            $\displaystyle d\langle X^{x, i}, X^{x, i'}\rangle_{s}
            = \sum_{j=1}^{m}\sigma_{ij}(X_{s}^{x})\sigma_{i'j}(X_{s}^{x})ds
            = (\sigma\sigma^{\ast})_{ii'}(X_{s}^{x})ds$
        \end{itemize}
        が成り立つ.

        \begin{screen}
            $\because)$
        \end{screen}
        
        これらより
        \begin{align}
            f(X_{t}^{x})
            &=
            \begin{multlined}[t]
                f(x)
                + \sum_{i=1}^{d}\int_{0}^{t}\frac{\partial f}{\partial x_{i}}(X_{s}^{x})\left\{\sum_{j=1}^{m}\sigma_{ij}(X_{s}^{x})dB_{s}^{j}+b_{i}(X_{s}^{x})ds\right\} \\
                + \frac{1}{2}\sum_{i, i'=1}^{d}\int_{0}^{t}\frac{\partial f^2}{\partial x_{i}\partial x_{i'}}(X_{s}^{x})\{(\sigma\sigma^{\ast})_{ii'}(X_{s}^{x})ds\}
            \end{multlined} \\
            &=
            \begin{multlined}[t]
                f(x)
                + \UB{\sum_{i=1}^{d}\sum_{j=1}^{m}\int_{0}^{t}\sigma_{ij}(X_{s}^{x})\frac{\partial f}{\partial x_{i}}(X_{s}^{x})dB_{s}^{j}}{=:M_{t}\text{ (CLM)}}
                + \sum_{i=1}^{d}\int_{0}^{t}b_{i}(X_{s}^{x})\frac{\partial f}{\partial x_{i}}(X_{s}^{x})ds \\
                + \frac{1}{2}\sum_{i, i'=1}^{d}\int_{0}^{t}(\sigma\sigma^{\ast})_{ii'}(X_{s}^{x})\frac{\partial f^2}{\partial x_{i}\partial x_{i'}}(X_{s}^{x})ds
            \end{multlined} \\
            &=
            \begin{multlined}[t]
                f(x)
                + M_{t}
                + \sum_{i=1}^{d}\int_{0}^{t}b_{i}(X_{s}^{x})\frac{\partial f}{\partial x_{i}}(X_{s}^{x})ds \\
                + \frac{1}{2}\sum_{i, i'=1}^{d}\int_{0}^{t}(\sigma\sigma^{\ast})_{ii'}(X_{s}^{x})\frac{\partial f^2}{\partial x_{i}\partial x_{i'}}(X_{s}^{x})ds.
            \end{multlined}
        \end{align}

        ここで
        \begin{align}
            g(x)
            := \frac{1}{2}\sum_{i, i'=1}^{d}(\sigma\sigma^{\ast})_{ii'}(x)\frac{\partial f^2}{\partial x_{i}\partial x_{i'}}(x)
            + \sum_{i=1}^{d}b_{i}(x)\frac{\partial f}{\partial x_{i}}(x)
        \end{align}
        と定めると,$g\in C_{0}(\mathbf{R}^d)\nazo.$
        
        \begin{screen}
            $\because)$
        \end{screen}

        よって
        \begin{align}
            M_{t}
            = f(X_{t}^{x})
            - f(x)
            - \int_{0}^{t}g(X_{s}^{x})ds
        \end{align}
        と表せるが,$f, g$: 有界より $M$ は有界 CLM であり,Proposition 4.7 (ii) から UIM であることまで言える.
        $M$ の martingale 性より
        \begin{align}
            E[f(X_{t}^{x})
            - f(x)
            - \int_{0}^{t}g(X_{s}^{x})ds]
            = 0.
        \end{align}

        一方,
        \begin{align}
            E[f(X_{t}^{x})
            - f(x)
            - \int_{0}^{t}g(X_{s}^{x})ds]
            &= E[f(X_{t}^{x})]
            - f(x)
            - \int_{0}^{t}E[g(X_{s}^{x})]ds \\
            &= Q_{t}f(x)
            - f(x)
            - \int_{0}^{t}Q_{s}g(x)ds.
        \end{align}
        \begin{align}
            \therefore 
            Q_{t}f(x) - f(x)
            = \int_{0}^{t}Q_{s}g(x)ds.
        \end{align}

        $(Q_{t})_{t\ge0}$ が Feller 半群であることから
        \begin{align}
            Lf
            = \lim_{t\to0}\frac{Q_{t}f - f}{t}
            = \lim_{t\to0}\frac{1}{t}\int_{0}^{t}Q_{s}gds
            = g.
        \end{align}

        したがって $f\in D(L)$ かつ $Lf=g$ が言えた.
    \end{enumerate}
\end{proof}

\end{document}
