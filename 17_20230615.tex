\documentclass{jsarticle}
% \documentclass[b4paper,landscape,14pt]{jsarticle}
\title{}
\author{}
\date{
% \number\year 年 \number\month 月
}
\usepackage{fenrir_v1_4_0}
\usepackage{ethm_v1_1_0}
\mathtoolsset{showonlyrefs=true}

\begin{document}
% \maketitle
\setcounter{section}{7}
\section{Stochastic Differential Equations}
\setcounter{subsection}{2}
\subsection{Solutions of Stochastic Differential Equations as Markov Processes}

この節では,$\sigma(t, y)=\sigma(y), b(t, y)=b(y)$ として同様の場合を考える.
前の節では $\sigma, b$ に対し Lipschitz 条件,つまりある $K\ge0$ が存在し,任意の $x, y\in\real^d$ に対して
\begin{align}
    \lvert \sigma(x)-\sigma(y)\rvert \le K\lvert x-y\rvert,\quad
    \lvert b(x)-b(y)\rvert \le K\lvert x-y\rvert
\end{align}
が成り立つことを仮定していた.

$X^x$: $E_x(\sigma, b)$ の解とする ($x\in\real^d$).
weak uniqueness が成り立つため,任意の $t\ge0$ に対し $X_t^x$ の分布は解の選び方に依らない\nazo.
実際,この分布は $C(\real_+, \real^d)$ 上の,写像 $w\mapsto F_x(w)_t$ の下での Wiener 測度の像である($F_x$: Theorem 8.5 で導入された写像).
また次の定理では $X_t^x$ の像が変数の組 $(x, t)$ に連続的に依存することを示す.

\setcounter{thm}{5}
\begin{screen}
    \begin{thm}
        $(X_t)_{t\ge0}$: (complete) filtered probability space $(\Omega, \mathcal{F}, (\mathcal{F}_t), P)$ 上の $E(\sigma, b)$ の解 \\
        $\implies (X_t)_{t\ge0}$: filtration $(\mathcal{F}_t)$ に関する Markov 過程であり,
        \begin{align}
            Q_{t}f(x)
            = E[f(X_t^x)]
        \end{align}
        の形で定まる半群 $(Q_t)_{t\ge0}$ が定義される($X^x$: $E_x(\sigma, b)$ の任意の解).
    \end{thm}
\end{screen}

\end{document}
